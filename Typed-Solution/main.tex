\documentclass{homework}

\title{Problem Set 8}
\author{Albert-Kenneth Okine}

% ========== Importing packages
\usepackage[linesnumbered,ruled]{algorithm2e}   % for algorithm formatting
\usepackage{algorithm2e}
\usepackage{setspace}
\setstretch{1.2}
\RestyleAlgo{ruled}
\usepackage{setspace}
\setstretch{1.2}
\usepackage{comment}                            % for algorithm comments
\usepackage{amsmath}                            % math related tools
\usepackage{lscape}
\usepackage{sectsty}
% ========== Importing Packages

\sectionfont{\fontsize{12}{12}\selectfont}
\subsectionfont{\fontsize{12}{12}\selectfont}
\counterwithin*{equation}{section}
\counterwithin*{equation}{subsection}

\begin{document}
\maketitle

\section*{References}
  \begin{itemize}
    \item \textit{Introduction to Algorithms}, fourth edition.
    \item Jonathon Wang for being my partner
  \end{itemize}

  \pagebreak

\section*{14-8a} % ============================================================
\textit{Suppose that you are given a color picture consiting of an $m*n$ array
  $A[1:m, 1:n]$ of pixels, where each pixel specifies a triple of red, green,
  and blue (RGB) intensities. You want to compress this picture slightly, by
  removing from each of the $m$ rows, so that the whole picture becomes one
  pixel narrower. To avoid possible incongruous visual effects, however, the
  pixels removed in two adjacent rows must lie in either the same column or
  adjacent columns. In this way, the pixels removed form a "seam" from the
  top to the bottom row where successive pixels in the seam are adjacent
  vertically or diagonally. Show that the number of such possible seams grows
  at least exponentially in $m$, assuming that $n > 1$}.
  
  Since we want to assume that the number of such possible seams is at least
  exponentially in $m$, assuming that $n > 1$, we consider the case where
  $n = 2$. In this case, there are $n = 2$ columns in the input color picture,
  meaning that any choice of pixel as the seam in the row will satisfy the same
  or adjacent column rule. For each pixel in row $i$, we can make one of two
  choices, removing pixel $A[i,1]$ or pixel $A[i,2]$. Thus, there are $2$
  choices per row, with $n = 2$ rows, meaning there are $2 * 2 \cdots 2 = 2^m$
  unqiue combinations per $m$ rows and $2^m$.

  Any other choice of $n > 1$ is an extension of this example, and we have
  proven this property for $n = 2$, so we have proven that the number of
  possible seams grows exponentially in $m$, specifically $2^m$.

\section*{Designing Algorithm} % ==============================================
\textit{Given the following energy function for a pixel, design an algorithm to
  compute the minumum value vertical seam. i.e. a set of pixels, one per row,
  where pixels in adjacent rows have to be within one column of each other with
  minimum total energy. Give the algorithm in psuedocode, give me a proof of
  correctness, justify the runtime of the algorithm.}

    \begin{equation}
      E(x,y) =
      \left\{
        \begin{array}{lr}
          1000, & \text{(x,y) is on the border of the image}\\
          \sqrt[]{\sum_{d \in \Delta} d^2}, & \text{otherwise}
        \end{array}
      \right\}
    \end{equation}

  

\end{document}